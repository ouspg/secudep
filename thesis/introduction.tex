% Introduction Chapter

\iffalse
FIXME: TODO REMOVE THIS LIST
\begin{itemize}
\item INTRODUCTION: The Setting - bird eye's view - the challenge to be tackled / thing to be be improved in general
\item INTRODUCTION: Past research done
\item INTRODUCTION: Gap in knowledge/problem not yet solved
\item INTRODUCTION: Purpose and method of this work
\item INTRODUCTION: More detailed description what was done
\item INTRODUCTION: Results acquired
\item INTRODUCTION: Analysis and limitations of the result (Mostly relocate to Conclusions)
\item INTRODUCTION: Value (Mostly relocate to Conclusions)
\end{itemize}
\fi

Loading operating system into computer remotely over network
(``network booting'', ``diskless booting'') has been used for
decades. Network booting can be used to bootstrap operating system
installation (``network installation'') or it could be used for
diskless nodes to load the operating system and run it using disk
provided by server.

Usually network installation systems are built to serve single
organization (e.g.\ university department or single business) to
achieve repeatable and homogeneous installations. For example school's
computer class wants to have identical installations on all machines
and reinstalling the machines should be as simple and as fast as
possible.

Many Linux distributions offer ``netinstall'' where small image is
used to boot the computer into state where rest of the installation
software and packages can be downloaded directly from Internet.

This thesis briefly identifies what network based threats there are
and then studies how to protect the installation process using readily
available tools to enable encryption and code signing. Proof of
concept implementation of network installation system is tested to see
how encryption and code signing can help secure the installation
process.

\section{Protocols}

Multiple procotols have been developed and used in combination to
allow booting using IP network. Early published standards include RARP
(``A Reverse Address Resolution Protocol'', RFC903, published
1984\cite{RFC903}) and BOOTP (``Bootstrap Protocol'', RFC951,
published 1985\cite{RFC951}) could be used to allow ``a diskless
client machine to discover its own IP address''\cite{RFC951}, TFTP
(``Trivial File Transfer Protocol'', RFC783, published
1981\cite{RFC783}) ``may be used to move files between machines on
different networks implementing UDP.''\cite{RFC783}.

Later developments include RARP and BOOTP to be superseded by DHCP
(``Dynamic Host Configuration Protocol'', RFC1531, published
1993\cite{RFC1531}) and TFTP superseded by NFS (``Network File
System'', RFC1094, published 1989\cite{RFC1094}) which ``provides
transparent remote access to shared files across
networks.''\cite{RFC1094} PXE (``Preboot Execution
Environment''\cite{PXEspec}) is specification from Intel Corporation
to standardize preboot environment for network booting.


\section{Current state}

Alpine Linux's PXE Boot HOWTO\cite{alpine-pxe-boot-howto} summarises
the current situation:

\begin{quote}
Alpine can be PXE booted starting with Alpine 2.6-rc2. In order to
accomplish this you must complete the following steps:

\begin{itemize}
\item Set up a DHCP server and configure it to support PXE boot.
\item Set up a TFTP server to serve the PXE bootloader.
\item Set up an HTTP server to serve the rest of the boot files.
\item Set up an NFS server from which Alpine can load kernel modules.
\item Configure mkinitfs to generate a PXE-bootable initrd.
\end{itemize}
\end{quote}

As we can see, the whole process still relies on old protocols DHCP,
TFTP, HTTP and NFS developed around 1980--1990. However, these
protocols are not secure and should not be used over Internet.

TFTP, NFS and HTTP protocols can be replaced with HTTPS (HTTP over
TLS) where TLS protocol provides communications security using
cryptocraphy and authentication of one or both communicating parties.


\section{Challenging environments}

Computer networks are not safe nor secure. Internet being the most
unsafe of networks. Connections in Internet doesn't see national
borders and travel through different legislations. It's passed from
Internet service provider to another. On every step of the connection
someone might be listening or even alterning the connection to ones
own agendas. It might be governmental body (like NSA's PRISM
program \cite{nsa-prism}), criminal organization who have gained
foothold on point of network or simply curious individual just being
able to do so.

Same problems can also be present in networks like corporate
intranets, university networks, etc. where both government and
criminal organizations might have gained foothold to operate. In
USENIX Enigma 2016 conference Rob Joyce, Chief of Tailored Access
Operations in National Security Agency \cite{nsa-tao} describes how
his team infiltrates networks and moves there laterally to gain what
they are after.



\section{Threats}

Tanenbaum's Computer Networks\cite{Tanenbaum} divides network security
threats into four categories: secrecy, authentication, nonrepudiation
and integrity control. Secrecy (or data confidentiality) means sender
of the message encrypts the content so only receiver with correct key
can decrypt the content and see the message. Authentication ensures
receiving, transimitting or both parties determine they are
communicating with intented party before exchanging any confidential
messages. Integrity control guarantees that message cannot be modified
during transfer. Nonrepudiation ensures proof of integrity and the
origin of data. This is usually achieved with using authentication and
integrity control.

Threats can be indentified in all components from hardware to
operating system vulnerabilities. Table~\ref{tab:threats_table} lists
some common known attacks which could be targeted towards network
booting or network installation infrastructure.

\begin{table}[!ht]
  % Add some padding to the table cells:
  \def\arraystretch{1.1}%
  \begin{center}
    \label{tab:threats_table}
    \begin{tabular}{| l | l | l |}
      \hline
      Component   & Role               & Threat(s)                  \\
      \hline
      HTTP        & File transfer      & secrecy, integrity control \\
      DNS         & Name service       & nonrepudiation             \\
      NFS         & File transfer      & secrecy, integrity control \\
      TFTP        & File transfer      & integrity control          \\
      DHCP        & Zero configuration & nonrepudiation             \\
      \hline
    \end{tabular}
    \caption{Roles and threats of various components used in operating
      system installation over network}
  \end{center}
\end{table}

DHCP and DNS protocols could be used to redirect (``hijack'') future
communications into malicious services. DHCP is commonly used to
assign IP address to client and give various information (TFTP
server's IP address, DNS servers' IP addresses). Malicious DHCP could
take over future TFTP and DNS communications. DNS has many uses, but
commonly it's used to translate host name into IP address. Malicious
DNS server could redirect future communications into malicious
services.

TFTP, NFS and HTTP protocols could be used to deliver malicious files
which when executed in target system compromise the operating system
installation or even infect the hardware the operation was performed
in.

There has been development to secure DHCP and DNS\@. That however
requires the network in question to be configured to take these
security measurements in action. But the threats can be detected by
other components (e.g.\ using TLS's server authentication, and code
signing) so there's no need to changes to network configuration. Thus
the installation can be done securely in any network and if something
malicious is detected the installation process can halted.

Hardware (e.g.\ physical server or laptop) and peripherals
(e.g.\ displays, keyboards, mice, removable medias) can have backdoored
firmware. The backdoors could have been installed already on factory
or firmware was infected with some malware previously ran on the
machine. Discussing mitigations for threats against hardware is out of
scope of this work.


\section{Mitigation}

Threats can be mitigated by using trusted media, secure communication
channel and cryptographically signed files.

Boot environment is loaded from trusted media, for example using
prebuilt USB mass media. This media contains software and files to
safely load next steps required to load operating system kernel and
other files safely over network.

Network communication is done using HTTPS with X.509 certificate
pinning. This authenticates the remote server and makes it harder to
MITM attack the connection. If secure channel can't be opened, the
boot process should be halted.

Signed files are used to ensure authenticity of files used for
booting. For example many Linux distribution mirrors only provide
files via HTTP or FTP servers which are susceptible to MITM attack. If
signature check fails the boot process should be halted.
