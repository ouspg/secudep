% Introduction Chapter

Loading an operating system into computer remotely over the network
(``network booting'', ``diskless booting'') has been used for
decades. Network booting can be used to bootstrap operating system
installation (``network installation'') or it can be used for diskless
nodes to load the operating system and to run it using a disk provided
by server~\cite{anvin2008x86}.

Usually network installation systems are built to serve a single
organization (e.g.\ a single business) inside their own networks
(business intranet) to achieve repeatable and homogeneous
installations. Preferably, the installation is expected to be as easy
and as fast as possible, and to require as little as possible human
interaction.

Many Linux distributions offer a ``net install'' which is a small
image used to boot the computer into a state where rest of the
installation software and packages can be downloaded directly from
Internet. After downloading files the installer is executed. It runs a
set of steps, and finally a fresh operating system installation is
completed on the computer.

This thesis has four parts. Introduction looks at the history and
current state of installation systems (how it has been done in the
past and how it works currently) and then identifies what network
based risks there are for an installation system. The next chapter
contains design principles for a proof of concept installation system
which tries to mitigate against the identified risks. Then the proof
of concept implementation is compared against another installation
system. Finally in the conclusions and discussion chapter the findings
are summarized and recommendations are given for new research and
development, and how the security of installation systems could be
improved further.

The purpose of this thesis is to identify network protocol based risks
in current installation systems and to show how cryptography can be
used to improve security and to eliminate network based risks.


\section{Information Security}

One definition of information security (InfoSec) is given in Finnish
legislation, Government Decree on Information Security in Central
Government (681/2010), which states that ``information security means
administrative, technical and other measures and arrangements to
comply with secrecy obligations and restrictions on use related to
information, as well as to ensure access to information and its
integrity and availability''~\cite{finlex-infosec-in-gov}.

The Finnish national security audit criteria (``Information security
audit tool for authorities'' or ``Katakri'' for short) further divides
information security into three separate divisions called security
management, physical security and information
assurance~\cite{katakri}.

Katakri's security management is about how information security should
be built-in into an organization from management down to each
individual person. Security management contains subjects like the
organization's security principles, security management tasks and
responsibilities, risk management, continuity and personnel security.

As the name implies, physical security consists of protecting
information from physical access. ``The aim of physical safeguards is
to deny surreptitious or forced entry by intruders, to deter, impede
and detect unauthorised actions and allow for segregation of personnel
in terms of access to Classified Information on the need-to-know
basis''~\cite{katakri}.

Katakri's information assurance category contains information security
requirements for electronic information.


\subsection{Information assurance}

Katakri's information assurance requirements are divided into four
separate sections: communications security, systems security, data
security and operations security.

Communications security is about computer networks, devices connected
to such networks and networks connected to other networks.

Systems security is about access controls, privileges and
authorizations when using computers and computer networks. Further
when using the systems, proper audit logging, protection against
unwanted software, and incident detection and recovery are required.

Data security is about keeping the secrets secret when the data is
either stored somewhere or moved from one place to another.

Operations security contains day to day tasks for managing the
information processing environment life cycle, for example change
management, backups and software vulnerability management. Operations
security also contain requirements for the handling and transfer of
classified information.


\subsection{This thesis in Information Security landscape}

Security of an installation system can be place into the information
assurance section of Katakri. Further, it can be categorized under
operations security. In operations security, the role of an
installation system is at the very beginning of information life cycle
management. A secure installation system takes care of setting up
an appropriate and properly secured operating system installation to
a computer which then can be trusted.

The implementation of an installation system needs to take care of the
requirements in communications, systems and data security to be able
to achieve a secure installation.

In this thesis it is assumed that proper management security and
physical security are already in place.


\section{Role of Humans}

Information security is not only a technical issue. Since humans
operate computers, networks and software, information security is also
a people problem~\cite{parsons2010human}\cite{anderson}. Roughly the
people problem can be divided into two categories: psychological
attacks against humans (social engineering) and ``getting things
done''.

Social engineering is an attack against human psychology and cognitive
biases. Attacks like phishing or pretexting are used to exploit human
weaknesses and the willingness to provide information like user names,
passwords or credit card data. A person under a social engineering
attack might think she is not providing anything sensitive or harmful,
but the attacker could use bits and pieces of information from
multiple attacks to gain whatever she is looking
for~\cite{greavu2014social}\cite{anderson}.

Another social engineering attack worth mentioning is tailgating or
piggybacking~\cite{fairbrother2014insider}. In case of successful
tailgating or piggybacking, the attacker gains physical access to
premises which can then lead to e.g.\ stealing of information or
assets like computers, physical keys, ID badges and money. Or, the
attacker might be planting hardware like a physical key logger,
listening device or USB (Universal Serial Bus) memory with malware.

In contrast to a social engineering attack against humans is the
``getting things done'' scenario where human is just trying to get
through the day without any intention to do anything malicious. An
employee has something which needs to be done, maybe under stress and
pressure, and technical information security measures like a pop-up
window alarming user about something or even a window asking for user
password, are a distraction from the task at hand. However, such
pop-ups are so common that humans are constantly used to click ``OK''
to continue without even reading nor thinking about the reason or
content of such notifications~\cite{anderson}.

In the case of installation systems, the ``getting things done'' is
the more probable information security issue. Imagine an integration
engineer with a tight schedule working in the customer's premises
trying to get computer systems up and running. Any problem or obstacle
increases stress and anxiety, and just ``getting things done'' without
worrying about information security risks becomes more likely.


\section{Involved Protocols}

% FIXME: Why RARP and BOOTP replaced with DHCP
% FIXME: Introduce protocol

Multiple network protocols have been developed and used to allow
booting using IP (Internet Protocol) networks. Early published
standards include RARP (``A Reverse Address Resolution Protocol'',
RFC903, published 1984~\cite{RFC903}) and BOOTP (``Bootstrap
Protocol'', RFC951, published 1985~\cite{RFC951}) which could be used
to allow ``a diskless client machine to discover its own IP
address''~\cite{RFC951}, and TFTP (``Trivial File Transfer Protocol'',
RFC783, published 1981~\cite{RFC783}) which ``may be used to move
files between machines on different networks implementing
UDP.''~\cite{RFC783}.

Later RARP and BOOTP were superseded by DHCP (``Dynamic Host
Configuration Protocol'', RFC1531, published 1993~\cite{RFC1531}),
while TFTP became superseded by NFS (``Network File System'', RFC1094,
published 1989~\cite{RFC1094}), which ``provides transparent remote
access to shared files across networks''~\cite{RFC1094}. PXE
(``Preboot Execution Environment''~\cite{PXEspec}) is a specification
from Intel Corporation to standardize the preboot environment for
network booting. In some cases TFTP or NFS or both can be replaced
with HTTP (``Hypertext Transfer
Protocol''~\cite{RFC1945}\cite{RFC2616}).


\section{Identifying Risks}

The CIA triad~\cite{cia-triad} divides network security into three
elements: confidentiality, integrity and availability~\cite{anderson}.

Confidentiality means that the sender of the message encodes the
content so that only a receiver can decode it and see the
message. Confidentiality can be achieved using encryption. Encryption
is the process of encoding a message using a secret key so that
decryption is only possible with the correct secret key. Keys used for
encryption and decryption might not be the same. Confidentiality can
be achieved in a network for example by using the TLS
protocol~\cite{RFC5246} to encrypt network traffic.

Integrity control guarantees that a message cannot be modified during
transfer. It consist of two parts: Non-repudiation and authenticity.

Non-repudiation ensures proof of integrity and the origin of
data. This is usually achieved with using authentication and integrity
control. Digital signatures~\cite{Diffie2006}\cite{Goldwasser1988} can
provide non-repudiation. Standards such as S/MIME~\cite{RFC5751} or
OpenPGP~\cite{RFC4880} can be used for digital signatures.

Authenticity ensures that the receiver, transmitter or both parties
determine they are communicating with another intended party before
exchanging any confidential messages. The TLS protocol provides means
to verify the authenticity of communicating parties~\cite{RFC5246}.

Availability means that systems are up and operational. Perfect
security could be achieved by turning everything off, but such systems
cannot be used. It is important to ensure availability so that network
services can be used. Availability can be achieved by allocating
enough human and computing resources to operate the services such as
facilities, computer systems and networks.

Risks can be identified in all components from hardware to operating
system vulnerabilities. Table~\ref{tab:risks_table} lists some
common known attacks which could be targeted towards network booting
or network installation systems.

All components in Table~\ref{tab:risks_table} are susceptible to
issues with availability. For example, if the network or one or more
components are not available, the whole stack of components is
inoperative.

\begin{table}[!ht]
  % Add some padding to the table cells:
  \def\arraystretch{1.1}%
  \begin{center}
    \caption{Roles and risks of various components used in operating
      system installations over network\label{tab:risks_table}}
    \begin{tabular}{| l | l | l |}
      \hline
      Component   & Role               & Risks                      \\
      \hline
      HTTP        & File transfer      & confidentiality, integrity \\
      DNS         & Name service       & non-repudiation            \\
      NFS         & File transfer      & confidentiality, integrity \\
      TFTP        & File transfer      & confidentiality, integrity \\
      DHCP        & Address resolution & non-repudiation            \\
      \hline
    \end{tabular}
  \end{center}
\end{table}

% FIXME: hijack or mitm?

DHCP and DNS (Domain Name System, RFC1035~\cite{rfc1035}) protocols
could be used to redirect (``hijack'') future communications towards
malicious services~\cite{green2005dns}\cite{ornaghi2003man}.

DHCP is commonly used to assign an IP address (Internet Protocol
address) to a client and to give it various bits of information such
as the IP address of a TFTP or DNS server. A malicious DHCP server
could take over the following TFTP and DNS communications.

DNS has many uses, but commonly it is used to translate a host name
into an IP address. A malicious DNS server could redirect future
communications towards malicious services.

TFTP, NFS and HTTP are used to transfer files between a client and
server. A malicious or compromised file server could be used to
deliver malicious files to a client which upon execution could
compromise the client operating system installation or even infect the
hardware the operation was performed in.

There has been efforts to secure DHCP and DNS.\@ However, that
requires the network, clients and servers to be configured to take
these security measures into action. But the risks can be detected by
other components (e.g.\ using TLS's server authentication, and digital
signatures), so there is no need for changes to network
configuration. The installation can be done securely in any network
and if something malicious is detected, the installation process can
halted.

% FIXME: Backdoors?
% FIXME: eliminate hardware earlier from thesis

Hardware (e.g.\ a physical server or laptop) and peripherals (e.g.\
displays, keyboards, mice, removable medias) can have backdoored
firmware~\cite{swierczynski2016interdiction}. The backdoors could have
been installed already at the factory or the firmware has been
infected with some malware that has been previously executed on the
machine. Mitigations for risks against malicious hardware is out of
scope for this work.


\section{Current state}

Software deployment technologies~\cite{SoftDep}, securing virtual
machines~\cite{Garfinkel2005} as well as cloud computing security
challenges~\cite{Owens2010}\cite{Hashizume2013} have been widely
studied. However, the network installation of an operating system is
still much the same as in the 1980s, and it still forms the basis for
operating system installation.

Alpine Linux's PXE Boot HOWTO~\cite{alpine-pxe-boot-howto} summarizes
the current situation:

\begin{quote}
Alpine can be PXE booted starting with Alpine 2.6-rc2. In order to
accomplish this you must complete the following steps:

\begin{enumerate}
\item Set up a DHCP server and configure it to support PXE boot.
\item Set up a TFTP server to serve the PXE boot loader.
\item Set up an HTTP server to serve the rest of the boot files.
\item Set up an NFS server from which Alpine can load kernel modules.
\item Configure mkinitfs to generate a PXE-bootable initrd.
\end{enumerate}
\end{quote}

Alpine Linux's documentation was chosen as an example because of their
claim that it is ``for power users who appreciate security, simplicity
and resource efficiency''~\cite{alpine-about}. Similar setup is
required for other Linux distributions like Red
Hat~\cite{redhat-network-install}, and for Microsoft
Windows~\cite{windows-network-install}.

As can be seen, the whole process still relies on old protocols DHCP,
TFTP, HTTP and NFS developed around 1980--1990. However, these
protocols provide no security and should not be used in
networks.

TFTP, NFS and HTTP protocols could be replaced with HTTPS (HTTP over
TLS) where TLS (Transport Layer Security Protocol~\cite{RFC5246})
provides communications security using cryptography and authentication
of one or both communicating parties.

DHCP is the standard protocol to centrally manage IP addresses for
clients. It is difficult to replace, so its shortcomings need to be
countered with other means.

DNS is also a vital protocol to the Internet. It provides translation
from host names to IP addresses and back (and other name
services). DNS is also a standard protocol. Work to protect DNS
traffic has been done (DNSSEC, RFC4035~\cite{rfc4035}) and DNSSEC is
slowly getting a foothold to protect DNS communications. A risk for
DNS in installation systems arises from man-in-the-middle
attacks. Without DNSSEC it is possible to continue to use DNS and use
other means outside of DNS protocol to verify that DNS is working as
it should.

Shortcomings of all of these protocols and how to mitigate against the
risks are discussed later.

\section{Challenging environments}

Computer networks are not safe nor secure~\cite{beyondcorp}. Internet
is likely the lest safe among networks. Only one compromised device in
a network is required to make the whole network unsafe. Connections in
the Internet do not observe national borders and travel through
different areas of laws and regulations. Protocol packets are passed
from one Internet service provider to another. On every step of the
connection's path someone might be listening or even altering the
connection to their own agenda. It might be a governmental body (like
NSA's PRISM program~\cite{nsa-prism}), a criminal organization that
has gained a foothold somewhere in the network or simply a curious
individual just being able to do so.

Same problems can also be present in networks like business intranets
where both governmental and criminal organizations might have gained a
foothold to operate in. In USENIX Enigma 2016 conference Rob Joyce,
Chief of Tailored Access Operations in National Security Agency
describes how his team infiltrates networks and moves there laterally
to gain what they are after~\cite{nsa-tao}. Therefore, intranets
should be treated with same level of mistrust as Internet.


\section{Mitigation}

Risks can be mitigated by using trusted media, secure communication
channels and cryptographically signed files.

A boot environment should be loaded from trusted media, for example
using prebuilt USB mass media. This media contains software and files
to safely load next steps required to load the operating system kernel
and other files safely over the network.

Network communication is done using HTTPS with X.509 certificate
pinning. This authenticates the remote server and makes it harder to
perform man-in-the-middle attack against the connection. If a secure
channel cannot be established, the boot process should be halted.

Signed files are used to ensure the authenticity of files used for
booting. For example many Linux distribution mirrors only provide
files via HTTP or FTP servers which are susceptible to
man-in-the-middle attack. If the signature check fails, the boot
process should be halted.


\section{Comparison to virtualization and cloud}

Installing an operating system into a virtual machine (in a cloud or
other virtualization platform) enjoys many benefits compared to
installation to a physical hardware. Virtualization provides easier
``programmable'' access to every state of virtual machine installation
from setting up the machine itself and its parameters (like
processors, memory amount, disk space, network) to pre-building ready
operating system images (machine images) to be booted in the
cloud. This is called ``Infrastructure as Code'' or
IaC~\cite{spinellis}.

Infrastructure as Code can be achieved for example with tools like
Packer~\cite{packer} which can be used to build machine images, and
Terraform~\cite{terraform} to set up virtual machines and launch
machine images to produce running virtual machines. Both tools use a
simple description language where operations can be specified and then
run using the tool itself.

When building machine images, it is possible to have operating system
installation files on local disk so that no network access is
required. Many operating systems also provide means to download the
installation files beforehand and to verify their authenticity. Or in
the case of cloud environments, it is possible that the user can use
pre-made machine images provided by the cloud provider.

However, physical computers require physical access in order to be
able to plug in devices and cables, to turn power on and to control
the first stages of startup before the operating system is running.

There are remote management solutions like Intelligent Platform
Management Interface (IPMI), which make it possible to remotely
control physical hardware. IPMI is a specification of interfaces for
monitoring and controlling physical computer hardware remotely via
network. Using IPMI it is possible to instruct a computer to turn on
or off, and to control BIOS settings~\cite{ipmi-spec}. However, using
IPMI still requires physical connections to be made and IPMI to be
properly configured.
