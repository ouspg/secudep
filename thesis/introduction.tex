% Introduction Chapter

\iffalse
\begin{itemize}
\item FIXME: TODO REMOVE THIS LIST
\item INTRODUCTION: The Setting - bird eye's view - the challenge to be tackled / thing to be be improved in general
\item INTRODUCTION: Past research done
\item INTRODUCTION: Gap in knowledge/problem not yet solved
\item INTRODUCTION: Purpose and method of this work
\item INTRODUCTION: More detailed description what was done
\item INTRODUCTION: Results acquired
\item INTRODUCTION: Analysis and limitations of the result (Mostly relocate to Conclusions)
\item INTRODUCTION: Value (Mostly relocate to Conclusions)
\end{itemize}
\fi

Loading operating system into computer remotely over network
(``network booting'', ``diskless booting'') has been used for
decades. Network booting can be used to bootstrap operating system
installation (``network installation'') or it could be used for
diskless nodes to load the operating system and run it using disk
provided by server~\cite{anvin2008x86}.

Usually network installation systems are built to serve single
organization (e.g.\ single business) inside their own networks
(business intranet) to achieve repeatable and homogeneous
installations. Installation preferably is expected be as easy and as
fast as possible, and require a human interaction as little as
possible.

Many Linux distributions offer ``net install'' which is a small image
used to boot the computer into a state where rest of the installation
software and packages can be downloaded directly from Internet. After
downloading files the installer is executed. It runs a set of steps
and then finally a fresh operating system installation is completed on
the computer.

This thesis has four parts. Introduction looks at history and current
state of installation systems (how it has been done in the past and
how it works currently) and then identifies what network based risks
there are to installation system. The next chapter contains design
principles of proof of concept installation system which tries to
mitigate against the identified risks. Then the proof of concept
implementation is compared against another installation system.
Finally in conclusions and discussion chapter the findings are
summarized and recommendations are given for new research and
development, and how security of installation systems could be
improved further.

The work done is a practical hands on study of how installation
systems work and how to improve the security of these systems.

% FIXME: Limit thesis to software (not hardware, or management security)

\section{Information Security}

One definition of information security (InfoSec) is given in Finnish
legislation, Government Decree on Information Security in Central
Government (681/2010), which states that ``information security means
administrative, technical and other measures and arrangements to
comply with secrecy obligations and restrictions on use related to
information, as well as to ensure access to information and its
integrity and availability''~\cite{finlex-infosec-in-gov}.

The Finnish national security audit criteria (``Information security
audit tool for authorities'' or ``Katakri'' for short) further divides
information security into three separate divisions called security
management, physical security and information
assurance~\cite{katakri}.

Katakri's security management is about how information security should
be built-in into organization from management down to each individual
person. Security management contains subjects like organization's
security principles, security management tasks and responsibilities,
risk management, continuity and personnel security.

Physical security as the name implies consists of protecting
information from physical access. ``The aim of physical safeguards is
to deny surreptitious or forced entry by intruders, to deter, impede
and detect unauthorised actions and allow for segregation of personnel
in terms of access to Classified Information on the need-to-know
basis''~\cite{katakri}.

Katakri's Information assurance has information security requirements
for electronic information.


\subsection{Information assurance}

Katakri's Information assurance requirements are divided into four
separate sections: communications security, systems security, data
security and operations security.

Communications security is about computer networks, devices connected
to such networks and networks connected to other networks.

Systems security is about access controls, privileges and
authorizations when using computers and computer networks. Further
when using the systems, proper audit logging, protection against
unwanted software, and incident detection and recovery are required.

Data security is about keeping the secrets secret when the data is
either stored somewhere or moved from places to places.

Operations security contains day to day tasks for managing information
processing environment life cycle, for example change management,
backups and software vulnerability management. Operations security
also contain requirements for handling and transfer of classified
information.


\subsection{This thesis in Information Security landscape}

Security of installation system can be put under information assurance
section of Katakri. Further, it can be categorized under operations
security. In operations security, installation system's role is the
very beginnings of information life cycle management. Secure
installation system takes care of setting up appropriate and properly
secured operating system installation to computer which then can be
trusted.

Implementation of installation system need to take care of the
requirements in communications, systems and data security to be able
to achieve secure installation.

In this thesis it is assumed that proper management security and
physical security are already in place.


\section{Role of Humans}

Information security is not only a technical issue. Since humans
operate the computers, networks and software, information security is
also a people problem~\cite{parsons2010human}\cite{anderson}. Roughly
the people problem can be divided into two categories: psychological
attacks against human (social engineering) and ``getting things
done''.

Social engineering is attack against human psychology and cognitive
biases. Attacks like phishing or pretexting are used to exploit human
weaknesses and good willingness to gather information like user names,
passwords or credit card data. The person under social engineering
attack might think she is not providing anything sensitive or harmful,
but social engineering attacker could use bits and pieces of
information from multiple attacks to gain whatever she is
looking~\cite{greavu2014social}\cite{anderson}.

Another social engineering attack worth mentioning is tailgating or
piggybacking~\cite{fairbrother2014insider}. In case of successful
tailgating or piggybacking, the attacker gains physical access to
premises which can then lead to for example stealing of information or
assets like computers, physical keys, ID badges and money. Or the
attacker might be planting hardware like physical key logger, listening
device or USB (Universal Serial Bus) memory with malware.

Compared to social engineering attack against human is the ``getting
things done'' where human is just trying to get through the day
without any intention to do anything malicious. Employee has something
which needs to be done, maybe under stress and pressure, and technical
information security measures like pop up window alarming user about
something or even window asking for user password, are distraction
from the task at hand. Such pop ups are however so common that humans
are constantly practiced to click ``OK'' to continue without even
reading nor thinking about the reason or content of such
notifications~\cite{anderson}.

In case of installation system, the ``getting things done'' is the
more probable information security issue. Imagine integration engineer
with tight schedule working on customer's premises trying to get
computer systems up and running. Any problem or obstacle increases
stress and anxiety, and just ``getting things done'' without worrying
about information security risks increases.


\section{Involved Protocols}

% FIXME: Why RARP and BOOTP replaced with DHCP
% FIXME: Introduce protocol

Multiple network protocols have been developed and used to allow
booting using IP network. Early published standards include RARP (``A
Reverse Address Resolution Protocol'', RFC903, published
1984~\cite{RFC903}) and BOOTP (``Bootstrap Protocol'', RFC951,
published 1985~\cite{RFC951}) which could be used to allow ``a
diskless client machine to discover its own IP
address''~\cite{RFC951}, TFTP (``Trivial File Transfer Protocol'',
RFC783, published 1981~\cite{RFC783}) ``may be used to move files
between machines on different networks implementing
UDP.''~\cite{RFC783}.

Later developments include RARP and BOOTP to be superseded by DHCP
(``Dynamic Host Configuration Protocol'', RFC1531, published
1993~\cite{RFC1531}) and TFTP superseded by NFS (``Network File
System'', RFC1094, published 1989~\cite{RFC1094}) which ``provides
transparent remote access to shared files across
networks''~\cite{RFC1094}. PXE (``Preboot Execution
Environment''~\cite{PXEspec}) is specification from Intel Corporation
to standardize preboot environment for network booting. In some cases
TFTP or NFS or both can be replaced with HTTP (``Hypertext Transfer
Protocol''~\cite{RFC1945}\cite{RFC2616}).


\section{Identifying Risks}

CIA triad~\cite{cia-triad} divides network security into three
elements: confidentiality, integrity and availability~\cite{anderson}.

Confidentiality means that the sender of the message encodes the
content so only a receiver can decode it and see the
message. Confidentiality can be achieved using encryption. Encryption
is a process of encoding a message using secret key so that decryption
is only possible with the correct secret key. Keys used for encryption
and decryption might not be the same key. Confidentiality can be
achieved in network for example by using TLS protocol~\cite{RFC5246}
to encrypt network traffic.

Integrity control guarantees the message cannot be modified during
transfer. It consist two parts: Non-repudiation and authenticity.

Non-repudiation ensures proof of integrity and the origin of
data. This is usually achieved with using authentication and integrity
control. Digital signatures~\cite{Diffie2006}\cite{Goldwasser1988} can
provide non-repudiation. Standards such as S/MIME~\cite{RFC5751} or
OpenPGP~\cite{RFC4880} can be used for digital signature format.

Authenticity ensures receiving, transmitting or both parties determine
they are communicating with intended party before exchanging any
confidential messages. TLS protocol provides means to verify
authenticity of communicating parties~\cite{RFC5246}.

Availability means that the systems are up and operational. Perfect
security could be achieved by turning everything off, but there is no
usability in such systems. It is important to ensure availability so
that network services can be used. Availability can be achieved by
allocating enough human and computing resources to operate the
services such as facilities, computer systems and networks.

Risks can be identified in all components from hardware to operating
system vulnerabilities. Table~\ref{tab:risks_table} lists some
common known attacks which could be targeted towards network booting
or network installation systems.

All components in Table~\ref{tab:risks_table} are susceptible to the
issues with availability. For example, if the network or one or more
components are not available, the whole stack of components is
inoperative.

\begin{table}[!ht]
  % Add some padding to the table cells:
  \def\arraystretch{1.1}%
  \begin{center}
    \caption{Roles and risks of various components used in operating
      system installation over network\label{tab:risks_table}}
    \begin{tabular}{| l | l | l |}
      \hline
      Component   & Role               & Risks                      \\
      \hline
      HTTP        & File transfer      & confidentiality, integrity \\
      DNS         & Name service       & non-repudiation            \\
      NFS         & File transfer      & confidentiality, integrity \\
      TFTP        & File transfer      & confidentiality, integrity \\
      DHCP        & Address resolution & non-repudiation            \\
      \hline
    \end{tabular}
  \end{center}
\end{table}

% FIXME: hijack or mitm?

DHCP and DNS protocols could be used to redirect (``hijack'') future
communications into malicious
services~\cite{green2005dns}\cite{ornaghi2003man}.

DHCP is commonly used to assign an IP address to a client and give it
various bits of information like TFTP server's IP address and DNS
servers' IP addresses. Malicious DHCP server could take over the
following TFTP and DNS communications.

DNS has many uses, but commonly it is used to translate host name into
IP address. Malicious DNS server could redirect future communications
into malicious services.

TFTP, NFS and HTTP protocols are used to transfer files between client
and server. Malicious or compromised file server could be used to
deliver malicious files to client which when executed in client system
compromise the operating system installation or even infect the
hardware the operation was performed in.

There has been development to secure DHCP and DNS.\@ That however
requires the network, clients and servers to be configured to take
these security measurements in action. But the risks can be detected
by other components (e.g.\ using TLS's server authentication, and
digital signatures) so there is no need to changes to network
configuration. Thus the installation can be done securely in any
network and if something malicious is detected the installation
process can halted.

% FIXME: Backdoors?
% FIXME: eliminate hardware earlier from thesis

Hardware (e.g.\ physical server or laptop) and peripherals (e.g.\
displays, keyboards, mice, removable medias) can have backdoored
firmware~\cite{swierczynski2016interdiction}. The backdoors could have
been installed already on factory or firmware was infected with some
malware previously ran on the machine. Mitigations for risks against
hardware is out of scope of this work.


\section{Current state}

Software deployment technologies~\cite{SoftDep}, securing virtual
machines~\cite{Garfinkel2005} as well as cloud computing security
challenges~\cite{Owens2010}\cite{Hashizume2013} have been widely
studied. However, network installation of operating system is still
much the same as in the 1980s and it is the base for operating system installation.

Alpine Linux's PXE Boot HOWTO~\cite{alpine-pxe-boot-howto} summarizes
the current situation:

\begin{quote}
Alpine can be PXE booted starting with Alpine 2.6-rc2. In order to
accomplish this you must complete the following steps:

\begin{enumerate}
\item Set up a DHCP server and configure it to support PXE boot.
\item Set up a TFTP server to serve the PXE boot loader.
\item Set up an HTTP server to serve the rest of the boot files.
\item Set up an NFS server from which Alpine can load kernel modules.
\item Configure mkinitfs to generate a PXE-bootable initrd.
\end{enumerate}
\end{quote}

Alpine Linux's documentation was chosen as an example because of their
claim that it is ``for power users who appreciate security, simplicity
and resource efficiency''~\cite{alpine-about}. Similar setup is
required for other Linux distributions like Red
Hat~\cite{redhat-network-install}, and for Microsoft
Windows~\cite{windows-network-install}.

As can be seen, the whole process still relies on old protocols DHCP,
TFTP, HTTP and NFS developed around 1980--1990. However, these
protocols provide no security and should not be used in
networks.

TFTP, NFS and HTTP protocols could be replaced with HTTPS (HTTP over
TLS) where TLS (Transport Layer Security Protocol~\cite{RFC5246})
provides communications security using cryptography and authentication
of one or both communicating parties.

DHCP is the standard protocol to centrally manage IP addresses for
clients. It is difficult to replace so it is shortcomings need to be
countered with other means.

Also DNS (Domain Name System, RFC1035~\cite{rfc1035}) is a vital
protocol to the internet. It provides translation from name to IP
addresses and back (and other name services). DNS is also a standard
protocol. Work to protect DNS traffic has been done (DNSSEC,
RFC4035~\cite{rfc4035}) and DNSSEC is slowly getting a foothold to
protect DNS communications. The risk for DNS in installation systems
is Man-in-the-Middle attack. Without DNSSEC it is possible to continue
using DNS and use other means outside of DNS protocol to verify DNS is
working as it should.

Shortcomings of all these protocols and how to mitigate against the
risks are discussed later.

\section{Challenging environments}

Computer networks are not safe nor secure~\cite{beyondcorp}. Internet
being the most unsafe of networks. It requires only one compromised
device in a network to make the whole network unsafe. Connections in
the internet do not see national borders and travel through different
areas of laws and regulations. Protocol packets are passed from one
internet service provider to another. On every step of the
connection's path someone might be listening or even altering the
connection to ones own agendas. It might be governmental body (like
NSA's PRISM program~\cite{nsa-prism}), criminal organization who have
gained foothold on point of network or simply curious individual just
being able to do so.

Same problems can also be present in networks like business intranets
where both governmental and criminal organizations might have gained
foothold to operate. In USENIX Enigma 2016 conference Rob Joyce, Chief
of Tailored Access Operations in National Security Agency describes
how his team infiltrates networks and moves there laterally to gain
what they are after~\cite{nsa-tao}. Therefore intranets should be
treated with same level of mistrust as internet.


\section{Mitigation}

Risks can be mitigated by using trusted media, secure communication
channels and cryptographically signed files.

Boot environment is loaded from trusted media, for example using
prebuilt USB mass media. This media contains software and files to
safely load next steps required to load operating system kernel and
other files safely over network.

Network communication is done using HTTPS with X.509 certificate
pinning. This authenticates the remote server and makes it harder to
Man-in-the-Middle attack the connection. If secure channel cannot be
established, the boot process should be halted.

Signed files are used to ensure authenticity of files used for
booting. For example many Linux distribution mirrors only provide
files via HTTP or FTP servers which are susceptible to
Man-in-the-Middle attack. If signature check fails the boot process
should be halted.


\section{Comparison to virtualization and cloud}

Installing operating system to virtual machine (in a cloud or other
virtualization platform) enjoys many benefits compared to installation
to a physical hardware. Virtualization gives easier ``programmable''
access to every state of virtual machine installation from setting up
the machine itself and it is parameters (like processors, memory
amount, disk space, network) to pre-building ready operating system
images (machine images) to be booted in the cloud. This is called
``Infrastructure as Code'' or IaC~\cite{spinellis}.

Infrastructure as Code can be achieved for example with tools like
Packer~\cite{packer} which can be used to build machine images, and
Terraform~\cite{terraform} to set up virtual machines and launch
machine images to produce running virtual machines. Both tools use
simple description language where operations can be specified and then
ran using the tool itself.

When building machine images it is possible to have operating system
installation files on local disk so no network access is
required. Also many operating systems provide means to download the
installation files beforehand and verify their authenticity. Or in
case of cloud, it is possible that the user can use pre made machine
images provided by the cloud provider.

Physical computer however require physical access to be able to plug
in devices and cables, turn power on and to control the first stages
of startup before operating system is running.

There are remote management solutions like Intelligent Platform
Management Interface (IPMI) which make it possible to remotely control
physical hardware. Intelligent Platform Management Interface (IPMI) is
a specification of interfaces for monitoring and controlling physical
computer hardware remotely via network. Using IPMI it is possible to
command computer to turn on or off, and to control BIOS
settings~\cite{ipmi-spec}. However, using IPMI still requires the
physical connections to be made and proper configuration of IPMI.

