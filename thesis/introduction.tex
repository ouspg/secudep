% Introduction Chapter

FIXME: TODO REMOVE THIS LIST
\begin{itemize}
\item INTRODUCTION: The Setting - bird eye's view - the challenge to be tackled / thing to be be improved in general
\item INTRODUCTION: Past research done
\item INTRODUCTION: Gap in knowledge/problem not yet solved
\item INTRODUCTION: Purpose and method of this work
\item INTRODUCTION: More detailed description what was done
\item INTRODUCTION: Results acquired
\item INTRODUCTION: Analysis and limitations of the result (Mostly relocate to Conclusions)
\item INTRODUCTION: Value (Mostly relocate to Conclusions)
\end{itemize}


\section{History}

Loading operating system into computer remotely (``network booting'',
``diskless booting'') over network became possible when networks
developed. Network booting could be used to bootstrap operating system
installation or it could be used for diskless nodes to load the
operating system and run it using disk provided by server.

During that time multiple procotols were developed and used in
combination to allow booting using UDP/IP network. RARP (``A Reverse
Address Resolution Protocol'', RFC903, published 1984\cite{RFC903}) or
BOOTP (``Bootstrap Protocol'', RFC951, published 1985\cite{RFC951})
could be used to allow ``a diskless client machine to discover its own
IP address''\cite{RFC951}, TFTP (``Trivial File Transfer Protocol'',
RFC783, published 1981\cite{RFC783}) ``may be used to move files
between machines on different networks implementing
UDP.''\cite{RFC783}.

Later developments include RARP and BOOTP to be superseded by DHCP
(``Dynamic Host Configuration Protocol'', RFC1531, published
1993\cite{RFC1531}) and TFTP superseded by NFS (``Network File
System'', RFC1094, published 1989\cite{RFC1094}) which ``provides
transparent remote access to shared files across
networks.''\cite{RFC1094}


\section{Current state}

Alpine Linux's PXE Boot HOWTO\cite{alpine-pxe-boot-howto} summarises
the current situation:

\begin{quote}
Alpine can be PXE booted starting with Alpine 2.6-rc2. In order to
accomplish this you must complete the following steps:

\begin{itemize}
\item Set up a DHCP server and configure it to support PXE boot.
\item Set up a TFTP server to serve the PXE bootloader.
\item Set up an HTTP server to serve the rest of the boot files.
\item Set up an NFS server from which Alpine can load kernel modules.
\item Configure mkinitfs to generate a PXE-bootable initrd.
\end{itemize}
\end{quote}

As we can see, the whole process still relies on old protocols DHCP,
TFTP, HTTP and NFS developed around 1980--1990.

\section{Threats}
