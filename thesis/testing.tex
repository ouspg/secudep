% Testing Chapter

In this section we take a look at three different implementations of
installation infrastructure. First one is traditional and is designed
to be used inside corporate network, next is more modern
(boot.foo.sh~\cite{boot.foo.sh}) designed to be used over internet and
then last one (secudep) is the implementation discussed in this
thesis. The operating system used for testing installation is CentOS
Linux Distribution because it was found on all three systems. The
comparison focuses on protocols used to achieve the
installation. Summary of the protocols used in various steps of
installation process can be found from
Table~\ref{tab:comparison_table}.

\begin{table}[!ht]
  % Add some padding to the table cells:
  \def\arraystretch{1.1}%
  \begin{center}
    \label{tab:comparison_table}
    \begin{tabular}{| l | l | l | l |}
      \hline
      Step               & traditional & over internet & secudep    \\
      \hline
      Zero configuration & DHCP        & DHCP          & DHCP       \\
      Name resolution    & DNS         & DNS           & DNS        \\
      Boot menu          & TFTP        & HTTP          & HTTPS (FS) \\
      File signatures    & N/A         & N/A           & HTTPS      \\
      kernel and initrd  & TFTP        & HTTP          & HTTP (FS)  \\
      Kickstart          & NFS         & HTTP          & HTTPS      \\
      Installation files & NFS         & HTTP          & HTTP       \\
      \hline
    \end{tabular}
    \caption{Comparison between how three different installation
      infrastructures use protocols. FS in table means File Signing.}
  \end{center}
\end{table}

In all three systems the same protocols are used for zero
configuration (DHCP) and name resolution (DNS).

Boot menu is used to display choices of operating systems to be
installed. TFTP and HTTP are the protocols used in traditional and
over internet systems where secudep uses HTTP over TLS with code
signed files.

File signatures are cryptographically calculated proofs to verify
other files. Only secudep uses file signing and the signature files
are fetched over HTTPS.

Kernel and initrd are the files needed to launch Linux
installation. Here traditional system uses TFTP to serve these files,
but over internet and secudep systems use HTTP. HTTP is used because
the files are fetched from CentOS's official mirror over
internet. Secudep uses file signing to verify downloaded content.

After kernel and initrd are downloaded and file signatures are
verified the execution is handled to kernel. This means that secudep
can't provide file signatures to any following files. However, there's
still two important steps in installation process: kickstart file and
installation files.

Kickstart is CentOS specific file for automating unattended
installation. The kickstart file is downloaded by the initrd system so
secudep can't do file signature verification. However, secudep uses
HTTPS where traditional relies on NFS and over internet system uses
HTTP.

Installation files are downloaded and then installed into hard drive
to achieve the operating system installation. Operating system
installer is trusted to verify (e.g. CentOS uses GPG signatures) the
downloaded content before extracting the files into hard drive.
