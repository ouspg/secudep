% Implementation Chapter

To see if it is possible to use HTTPS and digital signatures, a simple
proof of concept implementation of an installation system
called \emph{secudep} was implemented. Source code can be found from
secudep's project site on
GitHub~\footnote{https://github.com/ouspg/secudep}.

Secudep's implementation has three main design principles: ease of
use, ease of deployment and security. Deploying a new installation
system should be easy so that it encourages building small, easy to
update and easy to maintain setups. Ease of deployment might also
attract people to develop new use cases and applications on top of an
already existing system. With the implemented solution there should be
no need to have a monolithic and centralized installation system, and
instead designs can shift more towards personal or per application
installation systems.

An installation system should help the end user achieve a fresh
installation of operating system as well as install applications as
easily, smoothly and as fast as possible. Most of the decisions
required for installation should be made beforehand and automatized as
much as feasible.

Security is a more difficult design principle to tackle. An
installation system should concentrate on selecting and enforcing safe
defaults, and to guide the user to make safe choices.

This proof of concept implementation uses public-key cryptography to
digitally sign files so that the authenticity of those files can be
verified. Regenerating new key material should also be encouraged when
updating signature files. This renders old installation systems
unusable and forces updating of installation media (for example USB
mass media).

As an example, one security design principle is to halt the
installation process when a security measure detects an anomaly. An
example of such an anomaly could be an active man-in-the-middle
attack. If the user is given a choice to continue, she usually does
so, probably without understanding or investigating what caused the
issue, and thus rendering the security measure useless and allowing
the attack.

This implementation borrows lots of ideas and lessons learned from
boot.foo.sh~\cite{boot-foo-sh} and from the installation system used
by Faculty of Information Technology and Electrical Engineering in
University of Oulu.

\section{Tools}

Secudep uses iPXE~\cite{iPXE} as network boot firmware.\@ iPXE is a
PXE~\cite{PXEspec} implementation with additional features such as
support for booting via HTTP~\cite{RFC2616} protocol. Support for
HTTPS can also be compiled in.\@

Secudep's iPXE binary build is done inside container using the
Docker~\cite{Docker} software containerization platform to achieve
repeatable builds with managed dependencies. Docker is a platform to
easily build operating system level virtualization~\cite{Soltesz2007}
containers. Instead of virtualizing the hardware like Xen or KVM,
containers use the operating system's namespaces to separate
containerized applications from each other. Docker is not mandatory
for producing the build.

Python programming language, bash shell scripts and OpenSSL are used
to build individual parts of the system.

\section{Setting up the installation system}

Setting up the installation system using secudep consists of the
following steps. After the list, all of the steps are explained further.

\begin{enumerate}
  \item Generate digital signing keys
  \item Collect HTTPS servers' X.509 certificates for public key pinning
  \item Build iPXE bootable media
  \item Write configuration file
  \item Generate contents for deployment
\end{enumerate}

Digital signing keys are generated when deploying the installation
system. A private key is used to produce the signatures and a public
key is embedded into the installation image.

The installation system is deployed to an existing HTTPS server. Thus
this server's X.509 certificate can be fetched and embedded into the
installation image. This is now the only X.509 certificate to be
trusted, and no other HTTPS server can be used.

After the digital signing keys have been generated and X.509
certificates have been fetched, it is possible to build the bootable
installation media. This installation media file is written e.g.\ to a
USB memory, and can be used to launch the operating system
installation in a computer.

A configuration file binds things together. It specifies the HTTPS
server, where the keys and certificates are, what operating systems
can be installed, and where the required files can be found.

After all other steps are done, the files to be deployed on the HTTPS
server can be generated. This step fetches all of the required files,
calculates digital signatures, creates various boot scripts, and
builds a directory structure which should then be mirrored on the HTTP
server.

Future work on secudep should simplify these steps even further.
Digital signing keys could be automatically generated if missing,
X.509 certificate collection could be automated based on secudep's
configuration file.\@ iPXE media build could also be done every time
contents for deployment are generated.

\section{Deploying}

Everything needed for the installation system to operate (from the
server side) is generated under one directory. This directory can then
be published on the HTTPS server. The URL for the installation system
files is configured in secudep's configuration file.

\section{Security}

Secudep makes it as easy as possible to use public key pinning for
HTTPS hosts and digital signatures to verify authenticity of files.

iPXE is configured to require trusted files. A file is trusted only
after its signature is verified successfully. This requirement cannot
be turned off once it is turned on.
