% Implementation Chapter

FIXME: Intro to Docker

To see if it's possible to use HTTPS and digital signatures, a simple
implementation of installation system called \emph{secudep} was
implemented. Source code can be found from secudep's project site on
GitHub~\cite{secudep}.

Secudep's implementation has three main design principles: ease of
use, ease of deploy and security. Deploying new installation system
should be easy so that it encourages building small, easy to update
and easy to maintain setups. Ease of deployment might also attract
developing new use cases and applications on top of already existing
system. With the implemented solution there should be no need to have
monolithic and centralized installation infrastructure, but designs
can shift more towards personal or per application installation
infrastructure.

Installation infrastructure should help end user achieve fresh
installation of operating system and applications as easily, smoothly
and as fast as possible. Most of the decisions required for achieving
installation should be made beforehand and automatized as much as
feasible.

Security is more difficult design principle to tackle. For the
installation infrastructure the concentration should be on selecting
safe defaults and guide user to make safe choices.

This implementation borrows lots of ideas and lessons learned from
boot.foo.sh\cite{boot-foo-sh} and from installation infrastructure
used by Faculty of Information Technology and Electrical Engineering
in University of Oulu. These two systems are also compared to this
implementation in next chapter.

\section{Tools}

Secudep uses iPXE~\cite{iPXE} as a network boot firmware. iPXE is
PXE~\cite{PXEspec} implementation with additional features such as
support for booting via HTTP~\cite{RFC2616} protocol. Support for
HTTPS can also be compiled in.\@ iPXE binary build is done using
Docker~\cite{Docker} (FIXME: explain setup) software containerization
platform to achieve repeatable builds with managed dependencies.

Python programming language, bash shell scripts and OpenSSL are used
to build individual parts of the system.

\section{Setting up installation system}

Setting up installation system using secudep has the following steps:

\begin{enumerate}
  \item Generate digital signing keys
  \item Collect HTTP servers' X.509 certificates for public key pinning (FIXME: explain why, how)
  \item Build iPXE bootable media
  \item Write configuration file
  \item Generate contents for deployment
\end{enumerate}

Future work on secudep should simplify these steps even further.
Digital signing keys could be automatically generated if missing,
X.509 certificate collection could be automated based on secudep's
configuration file.\@ iPXE media build could also be done every time
contents for deployment are generated.

\section{Deploying}

Everything needed for installation system to operate (from server
side) are generated under one directory. This directory can then be
published on HTTPS server. The URL for the installation system files
is configured in secudep's configuration file.

\section{Security}

Secudep make it as easy as possible to use public key pinning for
HTTPS hosts and digital signatures to verify authenticity of files.

iPXE is configured to require trusted files. File is trusted only
after it's signature is verified successfully. This requirement can't
be turned off once it's turned on.
