% Implementation Chapter

Implementation has three main design principles: ease of use, ease of
deploy and security. Deploying new installation infrastructure should
be easy so that it encouraces building small and easy to update
setups. Ease of deployment might also attract developing new uses and
applications on top of already existing system. With the implemented
solution there should be no need to have monolithic and centralized
installation infrastucture, but things can shift more towards personal
or per application installation infrastructure.

Installation infrastructure should help end user achieve fresh
installation of operating system and applications as easily, smoothly
and as fast as possible. Most of the decisions required for achieving
installation should be made beforehand and automatised as much as
feasible.

Security is more difficult design principle to tackle. For the
installation infrastructure the concentration should be on selecting
safe defaults and guide user to make safe choices.

This implementation borrows lots of ideas and lesson's learned from
boot.foo.sh\cite{boot-foo-sh}.

\section{Ease of use}

\begin{table}[!ht]
\label{list:config}
  \begin{adjustwidth}{-3.5cm}{}
    \begin{scriptsize}
\begin{verbatim}
[DEFAULT]
webroot = https://raw.githubusercontent.com/ouspg/secudep/master/boot/
destdir = /Users/oherrala/ouspg/secudep/boot
signkey = /Users/oherrala/ouspg/secudep/src/codesign/codesign.key
signcert = /Users/oherrala/ouspg/secudep/src/codesign/codesign.pem

[centos7]
name = CentOS 7
kernel = http://mirror.centos.org/centos/7/os/x86_64/isolinux/vmlinuz
initrd = http://mirror.centos.org/centos/7/os/x86_64/isolinux/initrd.img
params = text utf8 inst.ks=https://raw.githubusercontent.com/ouspg/secudep/master/boot/centos7.ks
\end{verbatim}
    \end{scriptsize}
  \end{adjustwidth}
  \begin{center}
    \caption{Sample config file used to build
        installation infrastructure}
  \end{center}
\end{table}

\section{Ease of deploy}

\section{Security}
