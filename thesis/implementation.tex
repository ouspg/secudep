% Implementation Chapter

Implementation has three main design principles: ease of use, ease of
deploy and security. Deploying new installation infrastructure should
be easy so that it encourages building small, easy to update and easy
to maintain setups. Ease of deployment might also attract developing
new use cases and applications on top of already existing system. With
the implemented solution there should be no need to have monolithic
and centralized installation infrastructure, but designs can shift
more towards personal or per application installation infrastructure.

Installation infrastructure should help end user achieve fresh
installation of operating system and applications as easily, smoothly
and as fast as possible. Most of the decisions required for achieving
installation should be made beforehand and automatized as much as
feasible.

Security is more difficult design principle to tackle. For the
installation infrastructure the concentration should be on selecting
safe defaults and guide user to make safe choices.

This implementation borrows lots of ideas and lessons learned from
boot.foo.sh\cite{boot-foo-sh} and from installation infrastructure
used by Faculty of Information Technology and Electrical Engineering
in University of Oulu. These two systems are also compared to this
implementation in next chapter.

\section{Ease of use}

\section{Ease of deploy}

\section{Security}
