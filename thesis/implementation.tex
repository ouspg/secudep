% Implementation Chapter

To see if it's possible to use HTTPS and digital signatures, a simple
proof of concept implementation of installation system
called \emph{secudep} was implemented. Source code can be found from
secudep's project site on
GitHub~\footnote{https://github.com/ouspg/secudep}.

Secudep's implementation has three main design principles: ease of
use, ease of deploy and security. Deploying new installation system
should be easy so that it encourages building small, easy to update
and easy to maintain setups. Ease of deployment might also attract
developing new use cases and applications on top of already existing
system. With the implemented solution there should be no need to have
monolithic and centralized installation system, but designs can shift
more towards personal or per application installation systems.

Installation system should help end user achieve fresh installation of
operating system and applications as easily, smoothly and as fast as
possible. Most of the decisions required for achieving installation
should be made beforehand and automatized as much as feasible.

Security is more difficult design principle to tackle. For the
installation system the concentration should be on selecting and
enforcing safe defaults, and guide user to make safe choices.

The proof of concept implementation uses public-key cryptography to
digitally sign files so that the authenticity of those files can be
verified. It should also be encouraged to regenerate new key material
when updating signature files. This renders old installation system
unusable and forces updating of installation media (for example USB
mass media).

One security design principle is, for example, to halt the
installation process when a security measure detects an anomaly. Such
anomaly could for example be an active Man in the Middle attack. If
user is given a choice to continue, she usually does so. Probably
without understanding or investigating what caused the issue, thus
rendering the security measure useless and allowing the attack.

This implementation borrows lots of ideas and lessons learned from
boot.foo.sh~\cite{boot-foo-sh} and from installation system
used by Faculty of Information Technology and Electrical Engineering
in University of Oulu.

\section{Tools}

Secudep uses iPXE~\cite{iPXE} as a network boot firmware.\@ iPXE is a
PXE~\cite{PXEspec} implementation with additional features such as
support for booting via HTTP~\cite{RFC2616} protocol. Support for
HTTPS can also be compiled in.\@

Secudep's iPXE binary build is done inside container using
Docker~\cite{Docker} software containerization platform to achieve
repeatable builds with managed dependencies. Docker is tool to easily
build operating system level virtualization~\cite{Soltesz2007}
containers. Instead of virtualizing the hardware like Xen or KVM,
containers use operating system's namespaces to separate containerized
applications from each other. Docker is not mandatory for producing
the build.

Python programming language, bash shell scripts and OpenSSL are used
to build individual parts of the system.

\section{Setting up installation system}

Setting up the installation system using secudep has the following
steps. After the list, all the steps are explained further.

\begin{enumerate}
  \item Generate digital signing keys
  \item Collect HTTPS servers' X.509 certificates for public key pinning
  \item Build iPXE bootable media
  \item Write configuration file
  \item Generate contents for deployment
\end{enumerate}

Digital signing keys are generated when deploying the installation
system. Private key is used to produce the signatures and
public key is embedded into the installation image.

The installation system is deployed to known HTTPS server. Thus this
server's X.509 certificate can be fetched and embedded into
installation image. This is now the only X.509 certificate to be
trusted and no other HTTPS server can be used.

When digital signing keys are generated and X.509 certificates are
fetched, it's possible to build the bootable installation media. This
installation media file is written for example to USB memory and can
be used to launch the operating system installation in a computer.

Configuration file binds things together. It specifies the HTTPS
server, where the keys and certificates are and what operating systems
can be installed and where the required files can be found.

After all other steps are done, the files to be deployed on HTTPS
server can be generated. This step fetches all required files,
calculates digital signatures and various boot scripts, and builds
directory structure which then should be mirrored on the HTTP server.

Future work on secudep should simplify these steps even further.
Digital signing keys could be automatically generated if missing,
X.509 certificate collection could be automated based on secudep's
configuration file.\@ iPXE media build could also be done every time
contents for deployment are generated.

\section{Deploying}

Everything needed for installation system to operate (from server
side) are generated under one directory. This directory can then be
published on HTTPS server. The URL for the installation system files
is configured in secudep's configuration file.

\section{Security}

Secudep make it as easy as possible to use public key pinning for
HTTPS hosts and digital signatures to verify authenticity of files.

iPXE is configured to require trusted files. File is trusted only
after it's signature is verified successfully. This requirement can't
be turned off once it's turned on.
