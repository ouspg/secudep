%
% Template for Department of Electrical and Information Engineering Diploma Thesis v1.1.2013
% Authors: Mika Korhonen (original author), Pekka Pietikäinen, Christian Wieser, Teemu Tokola and Juha Kylmänen.
% If you make any improvements to this template, please contact ouspg@ee.oulu.fi
%

\documentclass[a4paper,12pt,titlepage]{dithesis}
\usepackage[english,finnish]{babel}
\usepackage[utf8]{inputenc}
\usepackage[T1]{fontenc}
\usepackage{times}
\usepackage{tabularx}
\usepackage{graphicx}
\usepackage{float}
\usepackage{enumerate}
\usepackage{placeins}
\usepackage{fancybox}
\usepackage{verbatim}
\usepackage{longtable}
\usepackage{di}
\usepackage[hyphens]{url}
\usepackage{boxedminipage}
\usepackage{subfigure}
\usepackage{multirow}
% my additions
\usepackage{listings}
\usepackage{alltt}
\usepackage{changepage}   % for the adjustwidth environment
% For screenshots, etc.
\usepackage{graphicx}
\graphicspath{ {images/} }
% end of additions
\tolerance=500

%\usepackage[a4paper,margin=2.5cm,dvips]{geometry}
%\geometry{papersize={210mm,297mm}}
%dvipdf -sPAPERSIZE=a4

% The following code removes %-signs with URL:s longer than 72 chars
\begingroup
\makeatletter
\g@addto@macro{\UrlSpecials}{%
  \endlinechar=13 \catcode\endlinechar=12
  \do\%{\Url@percent}\do\^^M{\break}}
 \catcode13=12 %
 \gdef\Url@percent{\@ifnextchar^^M{\@gobble}{\mathbin{\mathchar`\%}}}%
\endgroup %

%\selectlanguage{finnish}


\otsikko{Turvallinen käyttöönotto haastavissa ympäristöissä}
\title{Secure Deployment Story for Challenging Environments}

\etunimi{Ossi}
\sukunimi{Herrala}
\valvoja{prof. Juha Röning}
\koulutusohjelma{electrical} % {information | electrical}
\vuosi{2017}
\tyo{Bachelor} % {Bachelor | Master}
\kieli{english} % {finnish | english}

\begin{document}

\begin{titlepage}
	%\vspace*{10 mm}
	\centering{\includegraphics*[width=0.5\textwidth]{uni_logo}\\}
	{{\small FACULTY OF INFORMATION TECHNOLOGY AND ELECTRICAL ENGINEERING}\\}
	\vspace{65 mm}
	{\textbf{\LARGE \getfirstname\ \getlastname}\\}
	\vspace{15 mm}
	{\textbf{\LARGE SECURE DEPLOYMENT STORY\\FOR CHALLENGING ENVIRONMENTS\\}}
	\vspace{70 mm}
	{\large {Bachelor's Thesis}\\}
	{\large {Degree Programme in Electrical Engineering}\\}
	{\large {Aug 2017}\\}
	%\maketitle
\end{titlepage}

\selectlanguage{english}

\begin{abstract}

\iffalse
\begin{itemize}
\item FIXME: remove this list
\item Background information (present tense)
\item Principal activity (past tense/present perfect tense)
\item Methodology (past tense)
\item Results (past tense)
\item Conclusions (present tense/tentative verbs/modal auxiliaries)
\end{itemize}
\fi

Network based automatic installation of an operating system is and has
been a crucial method to manage masses of computers. Protecting every
step of the installation is important. Ability to trust the
installation system to complete operating system installation safely
and to produce a secure installation is an important step in the
information security life cycle.

This thesis reviews past and current state of network based automatic
installation systems based on what protocols they use. Then it
continues to identify risks to those protocols and to study how the
situation could be improved by using cryptography with Transport Layer
Security (TLS) protocol and digital signatures. This thesis shows how
these two existing technologies can be used to provide a more secure
installation system.

A proof of concept implementation using TLS and digital signatures is
specified, developed and then compared to an already existing and
publicly available installation system. Finally the proof of concept
implementation is tested and shown to detect and prevent
man-in-the-middle attacks.

\keywords network protocol, operating system, installation system, network security
\end{abstract}

\selectlanguage{finnish}
\begin{tiivistelma}

Automaattinen käyttöjärjestelmän asennus verkon yli on ollut ja on
edelleen tärkeä tapa hallita isoja määriä tietokoneita. Jokaisen
asennusvaiheen suojaaminen on tärkeää. Mahdollisuus luottaa
asennusjärjestelmän toimivan turvallisesti ja tuottavan turvallisen
asennuksen on tärkeää tietoturvan elinkaaressa.

Tämä kandidaatintyö käy läpi miten asennusjärjestelmät ovat toimineet
ennen ja nykyisin sen perusteella mitä protokollia ne
käyttävät. Seuraavaksi tunnistetaan prokotolliin liittyviä riskejä ja
miten tilannetta voisi parantaa käyttämällä salausta (Transport Layer
Security, TLS) ja digitaalisia allekirjoituksia. Kandidaatintyö
osoittaa, että näitä kahta olemassaolevaa teknologiaa voi käyttää
tuottamaan turvallisemman asennusympäristön.

Asennusjärjestelmästä määriteltiin ja toteutettiin soveltuvuusselvitys
hyödyntäen TLS-protokollaa ja digitaalisia allekirjoituksia. Tätä
järjestelmää verrattiin julkisesti saatavilla olevaan
asennusjärjestelmään. Lopuksi toteutettua järjestelmää testataan ja
todetaan sen havaitsevan ja estävän välimieshyökkäykset.

\avainsanat verkkoprotokolla, käyttöjärjestelmä, asennusjärjestelmä, verkon tietoturva
\end{tiivistelma}

\selectlanguage{english}
%\selectlanguage{finnish}

\sisluettelo
%\tableofcontents

\otsake{FOREWORD}

\iffalse
Your personal background (in brief)
Your personal experiences or the circumstances that motivated you to write your dissertation (in brief)
The target group for which your dissertation was written
The division of labor (when the dissertation has been written by more than one person)
Acknowledgements to individuals and institutions who have helped you in the writing and checking of the dissertation
\fi

I have been writing software, and designing, building, operating and
tearing down Linux and UNIX systems for two decades. Installing the
operating system has been an important part of the life cycle for
Linux and UNIX systems.

This thesis was motivated by the need for modernizing operating system
installation over Internet. It is a big subject with lots of little
details to work on. This thesis is just a small but important part of
it.

I hope my work with this thesis can contribute back to everyone who
has made my life so much easier with their installation systems.

I would like to thank my supervisor Prof. Juha Röning for the
opportunity to do this thesis for Oulu University Secure Programming
Group (OUSPG). Huge thanks to Christian Wieser for continuously taking
the time to sit down with me for followup and pushing me forward.

Thanks to everyone who attended OUSPG Open
2016\footnote{https://github.com/ouspg/ouspg-open} and especially to
everyone who participated in the Thesis Review session, and read my
work and gave feedback. Truly appreciated.

Thanks to my dad and sister for taking the time to read my work and
giving valuable feedback. Hugs to my spouse Ella, who not only read
and commented on my work, but also endured the process of me writing
it while also finishing her master's degree.


%\allekirjoitus{Oulu, Finland \today}

\otsake{ABBREVIATIONS}

\setlongtables
\begin{longtable}[l]{p{3cm}p{0.7\textwidth}}
  % Add your abbreviations to abbreviations.tex
  LAN & local area network\\
IP & Internet protocol\\
GUI & graphical user interface\\
HTTP & hypertext transfer protocol\\
UDP & user datagram protocol\\
CPU & central processing unit \\
RAM & random-access memory\\
FTP & file transfer protocol\\
WWW & world wide web \\
API & application programming interface \\
NAT & network address translation \\
CLI & command line interface \\
DPI & dots per inch \\

\end{longtable}
\setcounter{table}{0}

%Johdanto
\chapter{Introduction}
\sivunumerot{}
% Introduction Chapter

FIXME: TODO REMOVE THIS LIST
\begin{itemize}
\item INTRODUCTION: The Setting - bird eye's view - the challenge to be tackled / thing to be be improved in general
\item INTRODUCTION: Past research done
\item INTRODUCTION: Gap in knowledge/problem not yet solved
\item INTRODUCTION: Purpose and method of this work
\item INTRODUCTION: More detailed description what was done
\item INTRODUCTION: Results acquired
\item INTRODUCTION: Analysis and limitations of the result (Mostly relocate to Conclusions)
\item INTRODUCTION: Value (Mostly relocate to Conclusions)
\end{itemize}


\section{History}

Loading operating system into computer remotely (``network booting'',
``diskless booting'') over network became possible when networks
developed. Network booting could be used to bootstrap operating system
installation or it could be used for diskless nodes to load the
operating system and run it using disk provided by server.

During that time multiple procotols were developed and used in
combination to allow booting using UDP/IP network. RARP (``A Reverse
Address Resolution Protocol'', RFC903, published 1984\cite{RFC903}) or
BOOTP (``Bootstrap Protocol'', RFC951, published 1985\cite{RFC951})
could be used to allow ``a diskless client machine to discover its own
IP address''\cite{RFC951}, TFTP (``Trivial File Transfer Protocol'',
RFC783, published 1981\cite{RFC783}) ``may be used to move files
between machines on different networks implementing
UDP.''\cite{RFC783}.

Later developments include RARP and BOOTP to be superseded by DHCP
(``Dynamic Host Configuration Protocol'', RFC1531, published
1993\cite{RFC1531}) and TFTP superseded by NFS (``Network File
System'', RFC1094, published 1989\cite{RFC1094}) which ``provides
transparent remote access to shared files across
networks.''\cite{RFC1094}


\section{Current state}

Alpine Linux's PXE Boot HOWTO\cite{alpine-pxe-boot-howto} summarises
the current situation:

\begin{quote}
Alpine can be PXE booted starting with Alpine 2.6-rc2. In order to
accomplish this you must complete the following steps:

\begin{itemize}
\item Set up a DHCP server and configure it to support PXE boot.
\item Set up a TFTP server to serve the PXE bootloader.
\item Set up an HTTP server to serve the rest of the boot files.
\item Set up an NFS server from which Alpine can load kernel modules.
\item Configure mkinitfs to generate a PXE-bootable initrd.
\end{itemize}
\end{quote}

As we can see, the whole process still relies on old protocols DHCP,
TFTP, HTTP and NFS developed around 1980--1990.

\section{Threats}
 % ./introduction.tex

\chapter{Implementing secudep}
% Implementation Chapter

Implementation has three main design principles: ease of use, ease of
deploy and security. Deploying new installation infrastructure should
be easy so that it encourages building small, easy to update and easy
to maintain setups. Ease of deployment might also attract developing
new use cases and applications on top of already existing system. With
the implemented solution there should be no need to have monolithic
and centralized installation infrastructure, but designs can shift
more towards personal or per application installation infrastructure.

Installation infrastructure should help end user achieve fresh
installation of operating system and applications as easily, smoothly
and as fast as possible. Most of the decisions required for achieving
installation should be made beforehand and automatized as much as
feasible.

Security is more difficult design principle to tackle. For the
installation infrastructure the concentration should be on selecting
safe defaults and guide user to make safe choices.

This implementation borrows lots of ideas and lessons learned from
boot.foo.sh\cite{boot-foo-sh} and from installation infrastructure
used by Faculty of Information Technology and Electrical Engineering
in University of Oulu. These two systems are also compared to this
implementation in next chapter.

\section{Ease of use}

Setting up installation system using secudep has the following steps:

\begin{enumerate}
  \item Generate file signing keys
  \item Collect HTTP servers' X.509 certificates for public key pinning
  \item Build iPXE bootable media
  \item Write configuration file
  \item Generate contents for deployment
\end{enumerate}

Future work on secudep should simplify these steps even further. File
signing keys could be automatically generated if missing, X.509
certificate collection could be automated based on secudep's config
file. iPXE media build could also be done every time contents for
deployment are generated.

\section{Ease of deploy}

Everything needed for installation system to operate (from server
side) are generated under one directory. This directory can then be
published on HTTPS server. The URL for the installation system files
is configured in secudep's configuration file.

\section{Security}

Secudep tries to make it as easy as possible to use public key pinning
for HTTPS hosts and file signing to verify authenticity of files.

iPXE is configured to require trusted files. File is trusted only
after it's signature is verified successfully. This requirement can't
be turned off once it's turned on.
  % ./implementation.tex

\chapter{Case studies}
% Testing Chapter

\section{Method}

I studied three different implementations of installation
infrastructure (later called ``systems'') by installing operating
system using it and studying what protocols are used. Protocols were
identified from network traffic capture using Wireshark network
protocol analyzer. The operating system used for testing installation
was CentOS Linux Distribution because it was available on all three
systems.

First system was what I want to call ``traditional'' and is designed
to be used inside corporate network. Second system is a more modern
(boot.foo.sh~\cite{boot-foo-sh}) designed to be used over the internet
and then last one (secudep) was the implementation discussed in this
thesis. The comparison analyzes protocols used to achieve the
installation and doesn't go deeply into contents of the protocol
messages and only cursorily looks at how protocols are used.

All three installations were done using VirtualBox virtual
machines. VirtualBox allows network traffic to be captured from the
beginning of virtual machine's lifetime. This allowed me to capture
and study what happens with earliest stages of boot process.

\section{Result}

\begin{table}[!ht]
  % Add some padding to the table cells:
  \def\arraystretch{1.1}%
  \begin{center}
    \begin{tabular}{| l | l | l | l |}
      \hline
      Step               & traditional & over internet & secudep    \\
      \hline
      Zero configuration & DHCP        & DHCP          & DHCP       \\
      Name resolution    & DNS         & DNS           & DNS        \\
      Boot menu          & TFTP        & HTTP          & HTTPS (DS) \\
      Digital signatures & N/A         & N/A           & HTTPS      \\
      kernel and initrd  & TFTP        & HTTP          & HTTP (DS)  \\
      Kickstart          & NFS         & HTTP          & HTTPS      \\
      Installation files & NFS (DS)    & HTTP (DS)     & HTTP (DS)  \\
      \hline
    \end{tabular}
    \caption{Comparison between how three different installation
      infrastructures use protocols. DS in table means Digital
      Signatures.}
    \label{tab:comparison_table}
  \end{center}
\end{table}

Summary of the protocols used in various steps of installation process
can be found from Table~\ref{tab:comparison_table}.

I identified and named common steps each system used to achieve the
installation. The steps were ``zero configuration'', ``name
resolution'', ``boot menu'', ``kernel and initrd'', ``kickstart'' and
``installation files''. Secudep also has additional step ``digital
signatures''.

``Zero configuration'' is the first step and it's purpose was to get
IP address and DNS server addresses for system to be installed.

``Name resolution'' is used to translate host names into IP address to
communicate with other servers.

\begin{figure}[h]
  \includegraphics[width=\textwidth]{bootfoosh-bootmenu.png}
  \caption{boot.foo.sh boot menu showing selection of operating systems.}
  \label{fig:bootmenu}
\end{figure}

``Boot menu'' is used to display choices of operating systems to be
installed. Example of boot menu can be seen in figure~\ref{fig:bootmenu}.

``kernel and initrd'' are the files needed to launch Linux
installation. These two files are downloaded over the internet and
then kernel is executed and it continues the boot process.

``Kickstart'' is CentOS specific file for automating unattended
installation. It's set of instructions downloaded and executed by the
installation process.

``Installation files'' are the contents of operating system to be
installed. The files are downloaded and extracted to hard drive to
achieve the installation.

``Digital signatures'' are cryptographically calculated proofs to
verify signed content (for example contents of a file). If the content
is changed, the signature check fails and user can be alerted about
the incident.

\section{Analysis}

In all three systems the same protocols are used for zero
configuration (DHCP) and name resolution (DNS). DHCP and DNS are de
facto standard protocols to achieve the task they solve so there's no
surprise here.

Boot menu is used to display choices of operating systems to be
installed. TFTP and HTTP are the protocols used in traditional and
over the internet systems. Both TFTP and HTTP protocols are
susceptible to Man in the Middle attack. Secudep uses HTTPS (HTTP over
TLS) with signed files.

\begin{figure}[h]
  \includegraphics[width=\textwidth]{verify-fail.png}
  \caption{Installation process is halted when digital signature
    verification fails.}
  \label{fig:verify-fail}
\end{figure}

Only secudep uses digital signatures and the signature files are
fetched over HTTPS. This is the step missing from the traditional and
over the internet installation systems. Figure~\ref{fig:verify-fail}
shows how secudep's process is halted when digital signature
verification fails. This failure is clear indication that something is
definitely wrong.

Kernel and initrd are the files needed to launch Linux
installation. Here traditional system uses TFTP to serve these files,
but over internet and secudep systems use HTTP. Again TFTP and HTTP
are susceptible to Man in the Middle attacks. TFTP is the de facto
standard to deliver these files inside intranet systems. HTTP is used
because the files are fetched from CentOS's official mirror over the
internet. Secudep uses digital signatures to verify downloaded content.

After kernel and initrd are downloaded and digital signatures are
verified the execution is handled to kernel. This means that secudep
can't provide digital signatures to any following files. However, there's
still two important steps in installation process: kickstart file and
installation files.

Kickstart is CentOS specific file for automating unattended
installation. The kickstart file is downloaded by the initrd system so
secudep can't do digital signature verification. However, secudep uses
HTTPS where traditional relies on NFS and over internet system uses
HTTP. Both NFS and HTTP are suspecting to Man in the Middle attacks.

Installation files are downloaded and then installed into hard drive
to achieve the operating system installation. Operating system
installer is usually trusted to verify digital signatures (e.g. CentOS
uses GPG signatures) the downloaded content before extracting the
files into hard drive. CentOS documentation~\cite{centos-gpg} states
that

\begin{quote}
Each stable RPM package that is published by CentOS Project is signed
with a GPG signature. By default, yum and the graphical update tools
will verify these signatures and refuse to install any packages that
are not signed, or have an incorrect signature. You should always
verify the signature of a package prior to installation. These
signatures ensure that the packages you install are what was produced
by the CentOS Project and have not been altered by any mirror or
website providing the packages.
\end{quote}

However, when initrd file is downloaded over insecure protocol or file
content is not verified against signature it's possible for malicious
third party to inject it's own GPG keys into initrd and point
installation system to malicious host serving the operating system
installation files and thus gain full control of the installed system.
  % ./testing.tex

% \chapter{Discussion and conclusion}
% % Conclusions and discussion Chapter

\begin{itemize}
\item CONCLUSIONS: reference to purpose of study
\item CONCLUSIONS: value of / reasons for the study
\item CONCLUSIONS: review of important findings / conclusions
\item CONCLUSIONS: comments, explanations or speculations about findings
\item CONCLUSIONS: limitations of study
\item CONCLUSIONS: implications of study or generalizations
\item CONCLUSIONS: recommendations for future or practical applications - USUALLY SKIPPED
\end{itemize}

This thesis took a look what network based threats could face
installation infrastructure and then studied what kind of means could
be used to protect the initial phases (before OS kernel took the
control of excecution) of installation process using encryption and
file signing.

Protecting every step of communications over networks is important and
protecting installation infrastructure is no exception. This thesis
has shown that it's possible to take a step further in a more secure
installation infrastructure by using two technologies: encryption and
file signatures.

More secure systems can be build step by step by combining simple
individual components without the need for designing a whole new
systems and techonologies. Replacing old components (like TFTP, NFS
and HTTP protocols) with new ones (HTTPS protocol) and increasing the
use of file signatures is a small steps to take for big benefits in
security.

Linux distributions and other open source operating systems use GPG or
other file signing methods to protect the installation packages from
outside tampering which is a really good and important thing to
do. Some Linux distributions also protect the package database
metadata with file signatures, but some distributions have that
functionality turned off by default. Maybe mirrors at some point could
take step forward and enable HTTPS so files like kernel and initrd,
and package database metadata could be securely downloaded?

The public key to verify file signatures is also embedded into initrd
file. Is the initrd file downloaded and verified so that the embedded
public key can be trusted by the installation process?

More testing and verification should be performed for iPXE and it's
TLS implementation and file signing capabilities. This was
intentionally left out from this thesis.
  % ./discussion.tex

\chapter{Conclusion}
% Conclusions Chapter

This thesis first reviewed what network based risks could face an
installation system and then studied what kind of means could protect
the initial phases (before the operating system kernel takes control
of execution) of the installation process using encryption and digital
signatures.

Protecting every step in network communications is important, and
protecting installation systems is no exception. This thesis has shown
that it is possible to take a step further towards more secure
installation systems by using two technologies: encryption and digital
signatures.

More secure systems can be built step by step by combining simple
individual components without the need for designing whole new systems
and technologies. Replacing old components (like TFTP, NFS and HTTP
protocols) with newer but already existing ones (HTTPS protocol) and
increasing the use of digital signatures are small steps to take in
order to gain a big benefit in security. Furthermore, when using HTTPS
it is possible to use different authentication schemes to hide
installation scripts (kickstart files, etc.) which otherwise would be
visible to the Internet.

Linux distributions and other open source operating systems use
OpenPGP or other digital signature methods to protect the installation
packages from outside tampering, which is a really good and important
thing to do. Some Linux distributions also protect the package
database metadata with digital signatures, but some have that
functionality turned off by default. Maybe mirrors at some point could
take a step forward and enable HTTPS so files like kernel and initrd,
and package database metadata could be securely downloaded?

The initrd file also contains a public key to verify digital
signatures. Is the initrd file downloaded and verified so that the
embedded public key can be trusted by the installation process?

More testing and verification should be performed for iPXE and
its TLS implementation and digital signature capabilities. This was
intentionally left out from this thesis.
  % ./conclusion.tex

\bibliographystyle{di}
\bibliography{di}
\end{document}
