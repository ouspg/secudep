% Conclusions Chapter

This thesis first reviewed what network based risks could face an
installation system and then studied what kind of means could protect
the initial phases (before the operating system kernel takes control
of execution) of the installation process using encryption and digital
signatures.

Protecting every step in network communications is important, and
protecting installation systems is no exception. This thesis has shown
that it is possible to take a step further towards more secure
installation systems by using two technologies: encryption and digital
signatures.

More secure systems can be built step by step by combining simple
individual components without the need for designing whole new systems
and technologies. Replacing old components (like TFTP, NFS and HTTP
protocols) with newer but already existing ones (HTTPS protocol) and
increasing the use of digital signatures are small steps to take in
order to gain a big benefit in security. Furthermore, when using HTTPS
it is possible to use different authentication schemes to hide
installation scripts (kickstart files, etc.) which otherwise would be
visible to the Internet.

Linux distributions and other open source operating systems use
OpenPGP or other digital signature methods to protect the installation
packages from outside tampering, which is a really good and important
thing to do. Some Linux distributions also protect the package
database metadata with digital signatures, but some have that
functionality turned off by default. Maybe mirrors at some point could
take a step forward and enable HTTPS so files like kernel and initrd,
and package database metadata could be securely downloaded?

The initrd file also contains a public key to verify digital
signatures. Is the initrd file downloaded and verified so that the
embedded public key can be trusted by the installation process?

More testing and verification should be performed for iPXE and
its TLS implementation and digital signature capabilities. This was
intentionally left out from this thesis.
