% Conclusions and discussion Chapter

\begin{itemize}
\item CONCLUSIONS: reference to purpose of study
\item CONCLUSIONS: value of / reasons for the study
\item CONCLUSIONS: review of important findings / conclusions
\item CONCLUSIONS: comments, explanations or speculations about findings
\item CONCLUSIONS: limitations of study
\item CONCLUSIONS: implications of study or generalizations
\item CONCLUSIONS: recommendations for future or practical applications - USUALLY SKIPPED
\end{itemize}

This thesis took a look what network based risks could face
installation infrastructure and then studied what kind of means could
be used to protect the initial phases (before operating system kernel
took the control of execution) of installation process using
encryption and digital signatures.

Protecting every step of communications over networks is important and
protecting installation infrastructure is no exception. This thesis
has shown that it's possible to take a step further in a more secure
installation infrastructure by using two technologies: encryption and
digital signatures.

More secure systems can be build step by step by combining simple
individual components without the need for designing a whole new
systems and technologies. Replacing old components (like TFTP, NFS and
HTTP protocols) with new ones (HTTPS protocol) and increasing the use
of digital signatures is a small steps to take for big benefits in
security. Also when using HTTPS it's possible to use HTTP's
authentication schemes to hide installation scripts (kickstart files,
etc.) which otherwise would be visible to internet.

Linux distributions and other open source operating systems use
OpenPGP or other digital signature methods to protect the installation
packages from outside tampering which is a really good and important
thing to do. Some Linux distributions also protect the package
database meta data with digital signatures, but some distributions
have that functionality turned off by default. Maybe mirrors at some
point could take step forward and enable HTTPS so files like kernel
and initrd, and package database meta data could be securely
downloaded?

The public key to verify digital signatures is also embedded into
initrd file. Is the initrd file downloaded and verified so that the
embedded public key can be trusted by the installation process?

More testing and verification should be performed for iPXE and it's
TLS implementation and digital signature capabilities. This was
intentionally left out from this thesis.
